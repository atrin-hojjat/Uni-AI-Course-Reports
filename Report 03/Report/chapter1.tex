\chapter{شرح مسئله و روند کار}

\section{مقدمه}
در این گزارش عملکرد الگوریتم‌های جستجوی محلی متفاوت (
\lr{Annealing Search, Hill-Climbing search, Local beam search, Genetic algorithm}
) را روی دو مسئله‌ی فروشنده‌ی دوره‌گرد(
\lr{TSP}
\LTRfootnote{\href{https://en.wikipedia.org/wiki/Travelling_salesman_problem}{Traveling Salesman problem}}
)
و  مسئله صدق‌پذیری دودویی )
\lr{3-sat}
\LTRfootnote{\href{https://en.wikipedia.org/wiki/Boolean_satisfiability_problem/}{Boolean Satisfiability Problem}}
)
 بررسی خواهیم کرد.
 هردوی این مسائل 
\lr{NP Complete}
میباشند و این الگوریتم ها صرفا تخمینی از جواب خواهند بود.

\section{کدها و خروجی‌های برنامه}
تمام کد‌ها و خروجی‌ها (از جمله کد
\verb;latex;
)
در
\href{https://github.com/atrin-hojjat/Uni-AI-Course-Reports/blob/main/Report\%2003/}{اینجا}
قابل مشاهده می‌باشد.

\chapter{توابع ابتدایی}
\section{تولید گراف تصادفی}
برای تولید گراف تصادفی از تابع زیر استفاده شده است.
\LTRfootnote{\url{https://github.com/atrin-hojjat/Uni-AI-Course-Reports/blob/main/Report\%2003/code/generators/__init__.py}}
در این تابع گرافی کامل با 
\verb;nodes;
راس تولید شده که هر یال آن عددی بین
\verb;min_weight;
و
\verb;max_weight;
می‌باشد.

\begin{latin}
\begin{python}
def gen_weighted_graph(nodes=30, wei_min=1, wei_max=100):

    graph = [[0 for j in range(nodes)] for i in range(nodes)]

    for i in range(nodes):
        for j in range(i):
            wei = math.floor(random.random() * (wei_max - wei_min)) + wei_min
            graph[i][j] = wei
            graph[j][i] = wei
    return graph

\end{python}
\end{latin}
\section{تولید \lr{\ttfamily{3-SAT}} تصادفی}
تابع 
\verb;gen_sat;
از بین تمام سه‌تای های
$x, y, z \in \{x, \neg x | x \in \{x_1, x_2, ... x_n\}\}$
هریک را به احتمال
$p$
انتخاب می‌کند.

برای چک کردن همه‌ی این سه‌تای‌ها یک متغیر 
$xi \in {-1, 1}$
نگه میداریم که در صورت صفر بودن آن فرض میکنیم که در آن ستایی منظور عکس i بوده است.

\begin{latin}
\begin{python}
def gen_sat(variables=20, p=0.2):
    sat = []
    
    for i in range(variables):
        for xi in range(-1, 2, 2):
            for j in range(i + 1):
                for xj in range(-1, 2, 2):
                    for k in range(j + 1):
                        for xk in range(-1, 2, 2):
                                if random.random() < p:
                                    clause = [xk * (k + 1), xj * (j + 1), xi * (i + 1)]
                                    sat.append(tuple(clause))

    return sat

\end{python}
\end{latin}
 
 \section{تصویر سازی}
 برای نمایش دادن نمودارهای برنامه از
\verb;matplotlib;
\LTRfootnote{\url{https://matplotlib.org/}}
و
\verb;seaborn;
\LTRfootnote{\url{https://seaborn.pydata.org/}}
 و برای گراف‌ها از
\verb;networkx;
\LTRfootnote{\url{https://networkx.org/documentation/stable/}}
استفاده شده.
\LTRfootnote{\url{https://github.com/atrin-hojjat/Uni-AI-Course-Reports/tree/main/Report\%2003/code/visualizers}}


\section{نمایش \lr{\ttfamily{TSP}}}

برای نمایش 
\verb;TSP;
از تابع زیر استفاده شده.
\LTRfootnote{\url{https://github.com/atrin-hojjat/Uni-AI-Course-Reports/tree/main/Report\%2003/code/visualizers/GraphVisualizers.py}}

\begin{latin}
\begin{python}
def visualize_tsp(graph, path=[], name="",
        output=None):
    plt.figure()
    plt.title(name)
    G = nx.Graph()
    color_map = []
    for i in range(len(graph)):
        if i in path:
            color_map.append("#66CC99")
        else:
            color_map.append("#112233")
    G.add_nodes_from([i for i in range(len(graph))])
    path_edges = [sorted((path[i - 1], path[i])) for i in range(len(path))]
    print(path)
    print(graph)
    print(path_edges)
    G.add_edges_from(
            [
                (i, j, {'weight':  graph[i][j], 
                    'color': '#112233' if (i, j) not in path_edges else '#FC575E'})
                for i in range(len(graph)) for j in range(len(graph[i])) if (i < j)
            ]
    )
    G.add_edge(2, 5, weight=3)
    for i in range(len(path)):
        G.edges[path[i - 1], path[i]]['color'] = 'red'
    #  G.add_edges_from(
    #          [
    #              (path[i - 1], path[i], {'weight': graph[path[i - 1]][path[i]]})
    #              for i in range(len(path))

    #          ],
    #          color="#FC575E"
    #  )
    pos = nx.shell_layout(G)
    edge_colors = nx.get_edge_attributes(G,'color').values()
    nx.draw_networkx_edges(G, pos, alpha=0.4, edge_color=edge_colors)
    nx.draw_networkx_nodes(G, pos, node_color=color_map, node_size=100)
    labels = nx.get_edge_attributes(G,'weight')
    nx.draw_networkx_edge_labels(G,pos,edge_labels=labels)
    if output:
        if not os.path.exists(os.path.dirname(os.path.join("./output", output, f"{name}.jpg"))):
            try: os.makedirs(os.path.dirname(os.path.join("./output", output, f"{name}.jpg")))
            except OSError as exc: # Guard against race condition
                if exc.errno != errno.EEXIST:
                    raise
        plt.savefig(os.path.join("./output", output, f"{name}.jpg"))
        matplotlib.rcParams.update({
            "pgf.texsystem": "pdflatex",
            'font.family': 'mononoki Nerd Font Mono',
            'text.usetex': True,
            'pgf.rcfonts': False,
        })
        plt.savefig(os.path.join("./output", output, f"{name}.pgf"))
    plt.show()



\end{python}
\end{latin}
 
 
 
 
 
\chapter{پیاده‌سازی \lr{\ttfamily{TSP}}}
برای پیاده سازی این الگوریتم در فایل
\verb;LoadTSP.py;
دو تابع
\verb;load_samples;
و
\verb;run_tests;
نوشته شده‌است.
تابع
\verb;load_samples;
یک گراف تصادفی درست کرده و خروجی هر چهار الگوریتم را روی آن نمایش می‌دهد و تابع
\verb;run_tests;
خروجی الگوریتم‌‌ها را با افزایش
$N$
بررسی می‌کند.
الگوریتم‌های پیاده شده در 
\href{https://github.com/atrin-hojjat/Uni-AI-Course-Reports/blob/main/Report\%2003/code/TSP/}{اینجا}
قابل مشاهده هستند.

\section{جواب بهینه}

با توجه به سرعت پایین زبان
\verb;python;
برای پیداکردن جواب بهینه از
\verb;C;
استفاده‌شد.
این الگوریتم در زمان
$O(n^2 . 2^n)$
و حافظه‌ی
$O(n.2^n)$
اجرا می‌شود.
\footnote{\href{https://www.geeksforgeeks.org/travelling-salesman-problem-set-1/}{شرح الگوریتم}}
\footnote{\href{https://github.com/atrin-hojjat/Uni-AI-Course-Reports/blob/main/Report\%2003/code/tests/efficient_solvers/tsp.c}{کد}}
برای اجرای این کد دستور زیر را اجرا کنید.


\lr{\ttfamily{./build tsp}}

\section{تخمین ارزش}
در پیاده‌سازی هر چهار الگوریتم، برای تخمین، نیاز به ماکسیمایز کردن یک تابع داریم. از آنجایی که هدف 
\verb;TSP;
حداقل سازی طول مسیر است، تابع مورد نظرمان را تفاضل مجموع یال‌ها و طول مسیر قرار دادیم.

\section{همسایگی}
دو دور را همسایه می‌گوییم اگر  یکی با جابجا کردن دو عضو متوالی به دیگری تبدیل شود.

\section{\lr{\ttfamily{Local Beam Search}}}
در این تابع تعداد نمونه‌ها 
$10$
و  حداکثر تکرار‌ها برای تست‌ها
$1000$
و برای نمونه‌ها
$10000$
قرار داده شده.
\section{\lr{\ttfamily{Annealing Search}}}
در این تابع 
تغییرات دما به فرم 
$T_{next}=0.98 T_{now}$
با حداقل دمای  $0.000001$ و دمای اولیه 1 ثبت شده‌اند.
در تست‌ها حداقل دما 
$0.001$
قرار داده شده.

\section{الگوریتم ژنتیکی}
در این الگوریتم سایز جمعیت ده و احتمال
\verb;Mutation;
برابر
$0.05$
قرار داده‌شده.
حداکثر تکرارها برای تست‌ها 
$1000$
و برای نمونه‌ها
$10000$
می‌باشد.


\subsection{تولید مثل}
برای تولید بچه‌ی دو مسیر، یک بازه‌ی متوالی از پدر را انتخاب کرده و رئوسی که نیامده‌اند را به‌ترتیب حضورشان در مادر قرار می‌دهیم.

\begin{latin}
\begin{python}
    def crossover(parent0, parent1):
        start, end = random.randrange(len(graph)), random.randrange(len(graph))
        if start > end:
            start, end = end, start
        child = [None] * len(graph)
        for i in range(start, end + 1):
            child[i] = parent0[i]
        ptr = 0
        for i in range(len(graph)):
            if child[i] != None:
                continue
            while parent1[ptr] in child:
                ptr += 1
            child[i] = parent1[ptr]
        return child

\end{python}
\end{latin}

\subsection{\lr{\ttfamily{Mutation}}}
\verb;Mutation;
را حاصل جابجای دو عضو متوالی تعریف می‌کنیم.

\begin{latin}
\begin{python}
     def mutate(cell):
        ind = random.randrange(len(graph))
        cell[ind], cell[ind - 1] = cell[ind - 1], cell[ind]
        return cell

\end{python}
\end{latin}

\section{خروجی‌های نمونه}
\insertfig{figures/Simulated Annealing Search.pgf}{\lr{Simulated Annealing on $N=20$}}{ANS_S_02_20}
\insertfig{figures/Genetic Algorithm.pgf}{\lr{Genetic Algorithm on $N=20$}}{ANS_S_02_20}
\insertfig{figures/Hill-Climbing.pgf}{\lr{Hill-Climbing Algorithm on $N=20$}}{ANS_S_02_20}
\insertfig{figures/Local Beam Search.pgf}{\lr{Local Beam Search on $N=20$}}{ANS_S_02_20}


\section{مقایسه‌ی نتایج}

\insertfig{figures/worse than best answer.pgf}{\lr{\% Worse than best answer}}{ANS_S_02_20}

به‌طور میانگین
\verb;Simulated Annealing Search;
بهترین نتیجه را داشت ولی در کل تفاوت چندانی بین الگوریتم‌ها نبود.
به لحاظ زمان همه تا حداکثر تعداد تکرار ها پیش می‌رفتند.

\chapter{پیاده‌سازی \lr{\ttfamily{3-SAT}}}
در اینجا
از آنجایی که هر مسئله‌ی
\verb;SAT;
می‌تواند به یک
\verb;3-SAT;
تبدیل‌شود،
\LTRfootnote{\url{https://en.wikipedia.org/wiki/Boolean_satisfiability_problem\#3-satisfiability}}
 صرفا مسئله‌ی
\verb;3-SAT;
بررسی می‌شود.
کد الگوریتم‌های این بخش در 
\href{https://github.com/atrin-hojjat/Uni-AI-Course-Reports/blob/main/Report\%2003/code/SAT/}{اینجا}
قابل دسترسی است.

در فایل
\verb;LoadSAT.py;
تابع 
\verb;load_samples;
نوشته شده که خروجی‌های الگوریتم‌ها را برای یک مسئله‌ی
\verb;SAT;
تصادفی محاسبه می‌کند.



\section{تخمین ارزش}
برای تخمین ارزش یک جواب، در هر چهار الگوریتم، از تعداد جملات صحیح در
\verb;SAT;
استفاده شده است.

\section{همسایگی}
دو موقعییت همسایه‌اند، اگر تنها یک متغییر در آنها مقدار متفاوتی داشته باشد.

\section{\lr{\ttfamily{Local Beam Search}}}
در این تابع تعداد نمونه‌ها 
$10$
و  حداکثر تکرار‌ها برای تست‌ها
$1000$
و برای نمونه‌ها
$10000$
قرار داده شده.
\section{\lr{\ttfamily{Annealing Search}}}
در این تابع 
تغییرات دما به فرم 
$T_{next}=0.98 T_{now}$
با حداقل دمای  $0.000001$ و دمای اولیه 1 ثبت شده‌اند.
در تست‌ها حداقل دما 
$0.001$
قرار داده شده.

\section{الگوریتم ژنتیکی}
در این الگوریتم سایز جمعیت ده و احتمال
\verb;Mutation;
برابر
$0.05$
قرار داده‌شده.
حداکثر تکرارها برای تست‌ها 
$1000$
و برای نمونه‌ها
$10000$
می‌باشد.


\subsection{تولید مثل}
در تولید مثل، از یک
\verb;prefix;
مادر و 
\verb;suffix;
معادل آن در پدر استفاده می‌کنیم.

\begin{latin}
\begin{python}
    def crossover(parent0, parent1):
        start, end = random.randrange(n), random.randrange(n)
        if start > end:
            start, end = end, start
        child = [None] * n
        for i in range(start, end + 1):
            child[i] = parent0[i]
        ptr = 0
        for i in range(start):
            child[i] = parent1[ptr]
        for i in range(end + 1, n):
            child[i] = parent1[ptr]
        return child


\end{python}
\end{latin}

\subsection{\lr{\ttfamily{Mutation}}}
\verb;Mutation;
را حاصل عوض کردن یک متغییر تعریف می‌کنیم.

\begin{latin}
\begin{python}
    def mutate(cell):
        ncell = list(cell)
        i = random.randrange(n)
        ncell[i] = not ncell[i]
        return ncell


\end{python}
\end{latin}
\section{مقایسه‌ی نتایج}
در اینجا نیز تفاوت چندانی بین روش‌های مختلف چه از نظر زمانی و چی از نظر درستی مشاهده نمیشود.


