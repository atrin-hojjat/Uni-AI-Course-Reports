\chapter{شرح مسئله و روند کار}

\section{مقدمه}
در این گزارش با استفاده از الگوریتم 
\verb;Minimax;
یک
\verb;agent;
برای بازی
\verb;Othello;
می‌سازیم.
پیاده سازی و کد در
\href{https://github.com/atrin-hojjat/Uni-AI-Course-Reports/blob/main/Report\%2004/}{اینجا}
قابل مشاهده می‌باشد.
برای پیاده سازی
\verb;engine;
بازی و رابط کاربری از 
\lr{Python}
و برای پیاده سازی الگوریتم از
\lr{C++}
استفاده شده است.

\section{روش اجرا}
برای اجرای برنامه نیاز به 
\verb;Python;
با حداقل ورژن 
$3.8$
و 
\lr{gcc/g++}
با
\lr{c++17}
دارید.

برای اجرای برنامه ابتدا نیاز به راه اندازی یک
\lr{Virtual environment}
برای 
\verb;Python;
است.
به این منظور در فولدر
\lr{Report 04/game}
دستورات زیر را اجرا کنید.

\lstset{language=bash}
\begin{latin}
\begin{lstlisting}
cd Report\\ 04/game
python3 -m venv venv
\end{lstlisting}
\end{latin}

برای 
\verb;activate;
کردن با توجه به سیتم عامل دستورات
\href{https://packaging.python.org/guides/installing-using-pip-and-virtual-environments/}{اینجا}
\footnote{برای جزئیات بیشتر به 
\href{https://docs.python.org/3/library/venv.html}{اینجا}
 مراجعه کنید
 }
را اجرا کنید.

برای نصب پیشنیاز ها دستورات زیر را اجرا کنید.

\begin{latin}
\begin{lstlisting}
python3 -m pip install -r requirements.txt
\end{lstlisting}
\end{latin}

برای اضافه کردن ماژول 
\verb;Minimax;
باید کد 
\lr{c++}
الگوریتم به ماژول 
\lr{Python}
تبدیل شود.

\begin{latin}
\begin{lstlisting}
cd agent/MiniMaxCompEngine
python3 setup.py build
python3 setup.py install
cd ../../
\end{lstlisting}
\end{latin}

سپس با اجرای فایل
\lr{test01.py}
قادر به اجرای برنامه خواهید بود.

\begin{latin}
\begin{lstlisting}
python3 test01.py
\end{lstlisting}
\end{latin}

\chapter{پیاده‌سازی رابط کاربری و \lr{\ttfamily{Engine}} بازی}
\section{\lr{\ttfamily{Engine}}}
کلاس
\verb;Engine;
مسئولیت نگهداری موقعییت بازی را بر عهده دارد.
اعزای این کلاس شامل
\lr{boardSize, state, hisotry, availableMoves}
و توابع
\lr{getAvailableMoves, makeMove}
می‌باشد.
\lr{state}
یک
\lr{dict}
شامل موقعیت کنونی بازی، امتیاز هر بازیکن، آخرین بازی کنی که حرکت کرده و  اطلاعات پایان بازی می‌باشد.
\lr{history}
یک لیست از موقعیت ها‌ی کذشته و حرکات انجام شده است.

\subsection{تابع \lr{\ttfamily{getAvailableMoves}}}

تابع 
\lr{getAvailableMoves()}
با حرکت روی تمام خانه‌های خالی، چک می‌کند که اگر بازی‌کن کنونی این نقطه را انتخاب کند رنگ حداقل یک خانه‌ی همسایه تغییر می‌کند یا نه.
سپس لیستی از حرکات ممکن بر‌میگرداند.

\begin{latin}
\begin{python}
    def getAvailableMoves(self):
        if self.availableMoves:
            return [tuple(i) for i in self.availableMoves]
        curPlayer = (PLAYERS.BLACK if 
                self.state['lastPlayer'] == PLAYERS.WHITE else PLAYERS.WHITE)
        availableMoves = []
        
        for i in range(self.boardSize):
            for j in range(self.boardSize):
                if self.state['board'][i][j] != PLAYERS.NONE:
                    continue
                # Horizontal
                ok = False
                for jp in range(j + 1, self.boardSize):
                    if self.state['board'][i][jp] == PLAYERS.NONE:
                        break
                    elif self.state['board'][i][jp] == curPlayer:
                        ok = jp > j + 1
                        break

                if ok:
                    availableMoves.append((i, j))
                    continue
                for jp in range(j - 1, -1, -1):
                    if self.state['board'][i][jp] == PLAYERS.NONE:
                        break
                    elif self.state['board'][i][jp] == curPlayer:
                        ok = jp < j - 1
                        break
                if ok:
                    availableMoves.append((i, j))
                    continue


                # Vertical
                for ip in range(i + 1, self.boardSize):
                    if self.state['board'][ip][j] == PLAYERS.NONE:
                        break
                    elif self.state['board'][ip][j] == curPlayer:
                        ok = ip > i + 1
                        break
                if ok:
                    availableMoves.append((i, j))
                    continue
                for ip in range(i - 1, -1, -1):
                    if self.state['board'][ip][j] == PLAYERS.NONE:
                        break
                    elif self.state['board'][ip][j] == curPlayer:
                        ok = ip < i - 1
                        break
                if ok:
                    availableMoves.append((i, j))
                    continue

                # Diagonal
                for idel in range(-1, 2, 2):
                    for jdel in range(-1, 2, 2):
                        for k in range(1, self.boardSize):
                            if i + idel * k >= self.boardSize or i + idel * k < 0:
                                break
                            if j + jdel * k >= self.boardSize or j + jdel * k < 0:
                                break
                            ip, jp = i + idel * k, j + jdel * k
                            if self.state['board'][ip][jp] == PLAYERS.NONE:
                                break
                            elif self.state['board'][ip][jp] == curPlayer:
                                ok = k > 1
                                break
                        if ok: break
                    if ok: break
                if ok:
                    availableMoves.append((i, j))
                    continue

        self.availableMoves = [tuple(i) for i in availableMoves]
        return availableMoves


\end{python}
\end{latin}

\subsection{تابع \lr{\ttfamily{makeMove}}}

تابع
\lr{makeMove}
اندیس از خروجی
\lr{getAvailableMoves}
در‌یافت می‌کند و آن حرکت را برای بازی‌کن کنونی انجام می‌دهد.
سپس  چک می‌کند که آیا بازی به اتمام رسیده و آیا رقیب حرکت مجازی دارد یا نه و
اگر حریف حرکت مجازی نداشت نوبت را به  همان‌بازیکن بر ‌میگرداند.

\begin{latin}
\begin{python}
    def makeMove(self, move):
        if self.state['gameEnded']:
            return False
        if move not in self.availableMoves:
            return False
        self.history.append({'move': move, 'board': self.state['board']})
        board = np.copy(self.state['board'])
        self.state['board'] = board
        curPlayer = (PLAYERS.BLACK if 
                self.state['lastPlayer'] == PLAYERS.WHITE else PLAYERS.WHITE)
        self.state['lastPlayer'] = curPlayer

        i, j = move
        board[i][j] = curPlayer
        for jp in range(j + 1, self.boardSize):
            if self.state['board'][i][jp] == PLAYERS.NONE:
                break
            elif self.state['board'][i][jp] == curPlayer:
                for t in range(j + 1, jp):
                    self.state['board'][i][t] = curPlayer
                break

        for jp in range(j - 1, -1, -1):
            if self.state['board'][i][jp] == PLAYERS.NONE:
                break
            elif self.state['board'][i][jp] == curPlayer:
                for t in range(j - 1, jp, -1):
                    self.state['board'][i][t] = curPlayer
                break


        # Vertical
        for ip in range(i + 1, self.boardSize):
            if self.state['board'][ip][j] == PLAYERS.NONE:
                break
            elif self.state['board'][ip][j] == curPlayer:
                for t in range(i + 1, ip):
                    self.state['board'][t][j] = curPlayer
                break
        for ip in range(i - 1, -1, -1):
            if self.state['board'][ip][j] == PLAYERS.NONE:
                break
            elif self.state['board'][ip][j] == curPlayer:
                for t in range(i - 1, ip, -1):
                    self.state['board'][t][j] = curPlayer
                ok = ip < i - 1
                break

        # Diagonal
        for idel in range(-1, 2, 2):
            for jdel in range(-1, 2, 2):
                for k in range(1, self.boardSize):
                    if i + idel * k >= self.boardSize or i + idel * k < 0:
                        break
                    if j + jdel * k >= self.boardSize or j + jdel * k < 0:
                        break
                    ip, jp = i + idel * k, j + jdel * k
                    if self.state['board'][ip][jp] == PLAYERS.NONE:
                        break
                    elif self.state['board'][ip][jp] == curPlayer:
                        for kp in range(1, k):
                            self.state['board'][i + idel * kp][j + jdel * kp] = curPlayer
                        break
        self.availableMoves = None
        t = self.getAvailableMoves()
        
        ongoing = False
        blackScore, whiteScore = 0, 0
        for ix, iy in np.ndindex((self.boardSize, self.boardSize)):
            if self.state['board'][ix, iy] == PLAYERS.NONE:
                ongoing = True
            elif self.state['board'][ix, iy] == PLAYERS.WHITE:
                whiteScore = whiteScore + 1
            elif self.state['board'][ix, iy] == PLAYERS.BLACK:
                blackScore = blackScore + 1
        self.state['score']['black'], self.state['score']['white'] = blackScore, whiteScore
        if not ongoing or blackScore == 0 or whiteScore == 0:
            self.state['gameEnded'] = True

        if ongoing and len(t) == 0:
            # Skip One Move
            curPlayer = (PLAYERS.BLACK if 
                    self.state['lastPlayer'] == PLAYERS.WHITE else PLAYERS.WHITE)

            self.history.append({'move': None, 'board': np.copy(self.state['board'])})
            self.state['lastPlayer'] = curPlayer

        return True

\end{python}
\end{latin}

\section{رابط کاربری \lr{\ttfamily{CLI}}}
برای پیاده سازی 
\lr{\ttfamily{CLI}}
\LTRfootnote{\url{https://github.com/atrin-hojjat/Uni-AI-Course-Reports/blob/main/Report\%2004/game/interface/CLI.py}}
از کتابخانه
\lr{\ttfamily{Rich}}
\LTRfootnote{\url{https://github.com/willmcgugan/rich}}
استفاده شده که مسئولیت نشان دادن جدول و خروجی برنامه است.
با اجرای این  کد، کاربر ابتدا باید سایز برد و دو 
\verb;agent;
انتخاب کند.
\verb;agent;
ها شامل کاربر و 
\verb;Minimax;
با عمق‌های متفاوت است.
سپس تا زمانی که بازی تمام نشده با توجه به موقعیت
\verb;engine;
به یک بازیکن اجازه‌ی حرکت می‌دهد.

\section{\lr{\ttfamily{Agent}} ها}
هر
\verb;Agent;
\LTRfootnote{\url{https://github.com/atrin-hojjat/Uni-AI-Course-Reports/tree/main/Report\%2004/game/agent}}
یک زیرکلاس
\verb;AbstractAgent;
\LTRfootnote{\url{https://github.com/atrin-hojjat/Uni-AI-Course-Reports/blob/main/Report\%2004/game/agent/Agent.py}}
با دو تابع
\lr{\ttfamily{\_\_init\_\_(self, color)}}
و 
\lr{\ttfamily{move(self, engine)}}
می‌باشد.
به فراخوانی تابع 
\verb;move;
،
یک اندیس از
\lr{\ttfamily{engine.getAvailableMoves}}
باید بازگردانده شود.

\begin{latin}
\begin{python}
class AbstractAgent:
    def __init__(self, color):
        self.color = color
    
    def move(self, engine):
        pass

\end{python}
\end{latin}

\subsection{\lr{\ttfamily{Human Agent}}}
کلاس
\lr{\ttfamily{Human Agent}}
\LTRfootnote{\url{https://github.com/atrin-hojjat/Uni-AI-Course-Reports/blob/main/Report\%2004/game/agent/Human.py}}
در هر حرکت یک
\lr{\ttfamily{prompt}}
به کاربر نشان داده  و شماره‌ی حرکت را از کاربر می‌پرسد.
حرکت‌های ممکن و شماره‌هایشان در جدول نشان داده‌می‌شود.

\begin{latin}
\begin{python}
from . import Agent
from rich.prompt import IntPrompt
from engine.OthelloEngine import PLAYERS

class HumanAgent(Agent.AbstractAgent):
    def __init__(self, color):
        self.color = color
        self.colorName = "Black" if PLAYERS.BLACK == color else "White"

    def move(self, engine):
        moveNo = IntPrompt.ask(f"{self.colorName} to move")
        return moveNo

\end{python}
\end{latin}

\subsection{\lr{\ttfamily{Minimax Agent}}}
کلاس
\lr{\ttfamily{MiniMaxAgent}}
\LTRfootnote{\url{https://github.com/atrin-hojjat/Uni-AI-Course-Reports/blob/main/Report\%2004/game/agent/MiniMax.py}}
در ابتدا رنگ و عمق لازم را دریافت می‌کند.
سپس برای هر حرکت موقعیت بازی را به  تابع
\lr{\ttfamily{calculateMiniMax}}
از ماژول
\lr{\ttfamily{MiniMax}}
که به زبان
\lr{\ttfamily{C++}}
نوشته شده،
ارسال می‌کند تا بهترین حرکت محاسبه شود.
موقعیت بازی باید فرمت
\lr{\ttfamily{int32\_t}}
در
\lr{\ttfamily{C++}}
داشته باشد تا اطلاعات درست
\lr{\ttfamily{parse}}
شوند.

\begin{latin}
\begin{python}
from . import Agent
from rich.prompt import IntPrompt
from engine.OthelloEngine import PLAYERS
import numpy as np
from numpy.ctypeslib import ndpointer
from MiniMax import calculateMiniMax

class MiniMaxAgent(Agent.AbstractAgent):
    depth = 7
    def __init__(self, color, depth=7):
        self.color = color
        self.colorName = "Black" if PLAYERS.BLACK == color else "White"
        self.depth = depth


    def move(self, engine):
        grid = np.copy(engine.state['board'].astype(np.int32))
        N = engine.boardSize

        result = calculateMiniMax(N, grid, self.color,
                self.depth)

        return engine.getAvailableMoves().index(result['move'])
        

\end{python}
\end{latin}

\subsection{رجیستر کردن \lr{\ttfamily{agent}}ها }

در فایل
\lr{\ttfamily{agent/\_\_init\_\_.py}}
\LTRfootnote{\url{https://github.com/atrin-hojjat/Uni-AI-Course-Reports/blob/main/Report\%2004/game/agent/__init__.py}}
یک لیست از 
\lr{\ttfamily{Agent}}
های موجود برای صحولت کار
\lr{\ttfamily{CLI}}
نگه داشته می‌شود.
این لیست شامل یک
\lr{\ttfamily{Human Agent}}
و 
یازده
\lr{\ttfamily{MiniMax Agent}}
با عمق‌های متفاوت می‌باشد.

\begin{latin}
\begin{python}
from . import Human, MiniMax


AGENTS = [
        ("Human", Human.HumanAgent), 
        ("MiniMax agent(depth 2)", lambda col: MiniMax.MiniMaxAgent(color=col,
            depth=2)),
        ("MiniMax agent(depth 3)", lambda col: MiniMax.MiniMaxAgent(color=col,
            depth=3)),
        ("MiniMax agent(depth 4)", lambda col: MiniMax.MiniMaxAgent(color=col,
            depth=4)),
        ("MiniMax agent(depth 5)", lambda col: MiniMax.MiniMaxAgent(color=col,
            depth=5)),
        ("MiniMax agent(depth 6)", lambda col: MiniMax.MiniMaxAgent(color=col,
            depth=6)),
        ("MiniMax agent(depth 7)", lambda col: MiniMax.MiniMaxAgent(color=col,
            depth=7)),
        ("MiniMax agent(depth 8)", lambda col: MiniMax.MiniMaxAgent(color=col,
            depth=8)),
        ("MiniMax agent(depth 9)", lambda col: MiniMax.MiniMaxAgent(color=col,
            depth=9)),
        ("MiniMax agent(depth 10)", lambda col: MiniMax.MiniMaxAgent(color=col,
            depth=10)),
        ("MiniMax agent(depth 11 - up to 10 seconds)", lambda col: MiniMax.MiniMaxAgent(color=col,
            depth=11)),
        ("MiniMax agent(depth 12 - up to 100 seconds)", lambda col: MiniMax.MiniMaxAgent(color=col,
            depth=12)),
        ]

\end{python}
\end{latin}


\chapter{پیاده‌سازی الگوریتم \lr{\ttfamily{Minimax}}}
به دلیل محدودیت سرعت پایتون، برای پیاده‌سازی الگوریتم
\lr{\ttfamily{MiniMax}}
\LTRfootnote{\url{https://github.com/atrin-hojjat/Uni-AI-Course-Reports/blob/main/Report\%2004/game/agent/MiniMaxCompEng/calc_minimax_eng.cpp}}
از 
\lr{\ttfamily{C++}}
استفاده شد.
الگوریتم با 
\lr{\ttfamily{Alpha-Beta Pruning}}
و
\lr{\ttfamily{Cutoff}}
پیاده‌سازی شده
و از 
\lr{\ttfamily{Zobrist Hashing}}
برای پیش‌گیری از محاسبه دوباره استفاده شده.

برای بررسی امتیاز یک موقعییت از اختلاف تعداد مهره‌های هر بازیکن استفاده‌شده.

\section{\lr{\ttfamily{Zobrist Hashing}}}
در پیاده سازی 
\lr{\ttfamily{Zobrist Hashing}}
از
\href{https://www.geeksforgeeks.org/minimax-algorithm-in-game-theory-set-5-zobrist-hashing/}{این مقاله}
استفاده شده.

\lstset{language=C++}
\begin{latin}
\begin{lstlisting}
unsigned long long int zobristTable[MaxN][MaxN][STATES];
...


void generateZobristValues(int N) {
	for(int i = 0; i < N; i++)
		for(int j = 0; j < N; j++)
			for(int s = 0; s < STATES; s++)
				zobristTable[i][j][s] = randomGenerator();
}

unsigned long long int generateHash(const int N, int **grid) {
	unsigned long long int val = 0;
	for(int i = 0; i < N; i++)
		for(int j = 0; j < N; j++)
			val ^= zobristTable[i][j][grid[i][j]];
	return val;
}

\end{lstlisting}
\end{latin}

به‌علاوه با تغییر هر خانه‌ی جدول به شکل زیر می‌توان هش جدید را به‌دست آورد.

\begin{latin}
\begin{lstlisting}
newHash ^= zobristTable[ip][jp][prvColor] ^ 
		zobristTable[ip][jp][curColor];

\end{lstlisting}
\end{latin}

\section{تغییر جدول پس از هر حرکت}
برای بررسی حالت جدول در صورتی که یک حرکت خواص انجام شود، می‌توان هر دفعه کپی جدیدی از جدول درست کرد و به تابع فرستاد. اما در این پیاده‌سازی از آنجایی که حدف
\verb;Multithreading;
نداشتیم، در هر مرحله حرکت را روی جدول پیاده کردیم و پس از بررسی آن حالت جرول را به فرم قبلی باز گرداندیم.
به این منظور یک پشته
\LTRfootnote{stack}
از تغییرات نگه می‌داریم و  در حین انجام هر حرکت، تغییرات را به این پشته اضافه می‌کنیم.
هر تغییر را به شکل زیر نشان می‌دهیم:

\begin{latin}
\begin{lstlisting}
struct ChangedSegment {
	int k, is, js, idir, jdir;
	unsigned long long int prvHash;
	int prvColor, curColor;
};
\end{lstlisting}
\end{latin}

که نشان می‌دهد بازه‌ی
$(is + idir, js + jdir)$
تا
$(is + k * idir, js + k * jdir)$
از
\verb;prvColor;
به
\verb;curColor;
تغییر کرده‌اند.
\verb;prvHash;
نشان دهنده‌ی هش جدول پیش از شروع تغییرات است.

در تابع 
\verb;makeMove;
ابتدا در هر هشت جحت تمام تغییرات ممکن بررسی می‌شود سپس در یک 
\verb;for;
تمام آنها به پشته اضافه و در جدول اعمال می‌شوند.

\begin{latin}
\begin{lstlisting}
stack<ChangedSegment> moves = stack<ChangedSegment>();
...

unsigned long long int makeMove(const int N, int **grid, int player, 
		pair<int, int> selectedMove, unsigned long long int curHash) {
	int i = selectedMove.first, j = selectedMove.second;
	if(!~i || !~j || i < 0 || i >= N || j < 0 || j >= N)
		return curHash;
	if(grid[i][j] != 0) {
		cout << i << " " << j << endl;
		cout << grid[i][j] << endl;
		cout << "FATAL ERROR WRONG START" << endl;
		exit(-1);
	}
	vector<ChangedSegment> changes;

	changes.push_back({1, i, j, 0, 0, curHash, 0, player});


	for(int dir = -1; dir < 2; dir += 2) {
		for(int jp = j + dir; jp > -1 && jp < N; jp += dir) {
			if(grid[i][jp] == 0) 
				break;
			if(grid[i][jp] == player) {
				if(abs(jp - j) > 1) {
					ChangedSegment chngd;
					if(jp < j) 
						chngd = {abs(jp - j) - 1, i, j, 0, -1, curHash, 3 ^ player, player};
					else
						chngd = {abs(jp - j) - 1, i, j, 0, 1, curHash, 3 ^ player, player};
					changes.push_back(chngd);
				}
				break;
			}
		}
		for(int ip = i + dir; ip > -1 && ip < N; ip += dir) {
			if(grid[ip][j] == 0) 
				break;
			if(grid[ip][j] == player) {
				if(abs(ip - i) > 1) {
					ChangedSegment chngd;
					if(i < ip)
						chngd = {abs(ip - i) - 1, i, j, 1, 0, curHash, 3 ^ player, player};
					else
						chngd = {abs(ip - i) - 1, i, j, -1, 0, curHash, 3 ^ player, player};
					changes.push_back(chngd);
				}
				break;
			}
		}
	}
	for(int id = -1; id < 2; id += 2) 
		for(int jd = -1; jd < 2; jd += 2) 
			for(int k = 1; k < N; k++) {
				if(i + id * k < 0 || i + id * k >= N)
					break;
				if(j + jd * k < 0 || j + jd * k >= N)
					break;

				int ip = i + id * k;
				int jp = j + jd * k;

				if(grid[ip][jp] == 0)
					break;
				if(grid[ip][jp] == player) {
					if(k > 1) {
						ChangedSegment chngd = {k - 1, i, j, id, jd, curHash, 3 ^ player, player};
						changes.push_back(chngd);
					}
					break;
				}
			}

	unsigned long long int newHash = curHash;
	
	for(auto change : changes) {
		moves.push(change);
		// int dist = max(abs(change.ie - change.is), abs(change.je - change.js));
		for(int k = 1; k <= change.k; k++) {
			int ip = change.is + k * change.idir;
			int jp = change.js + k * change.jdir;
			if(ip >= N || ip < 0 || jp >= N || jp < 0) {

				cout << "FATAL ERROR " << change.k << " " << change.is << " " << change.js << " - " << ip << " " << jp << endl;
				exit(-1);
			}

			newHash ^= zobristTable[ip][jp][change.prvColor] ^ 
				zobristTable[ip][jp][change.curColor];
			grid[ip][jp] = change.curColor;
		}
	}
	return newHash;
}

\end{lstlisting}
\end{latin}

برای باز گرداندن حرکات از تابع
\verb;undoMove;
استفاده می‌شود که جدول را به حالتی باز می‌گرداند که هش آن
\verb;desHash;
باشد.

\begin{latin}
\begin{lstlisting}
void undoMoves(const int N, int **grid, int player, unsigned long long int desHash) {
	while(moves.size() && moves.top().prvHash == desHash) {
		auto change = moves.top();
		moves.pop();
		
		// int dist = max(abs(change.ie - change.is), abs(change.je - change.js));
		for(int k = 1; k <= change.k; k++) {
			int ip = change.is + k * change.idir;
			int jp = change.js + k * change.jdir;

			grid[ip][jp] = change.prvColor;
		}
	}
}
\end{lstlisting}
\end{latin}


\section{نگه‌داری موقعییت‌ها‌ی دیده شده}
برای هر موقعییت دیده شده، بهترین نتیجه‌ی حاصل از آن به همراه عمق بررسی و بهترین حرکت نگه داشته می‌شود.
\begin{latin}
\begin{lstlisting}
struct PreCalcData {
	int value, depth;
	pair<int, int> bestMove;
};
...

map<unsigned long long int, PreCalcData> preCalc[STATES];

\end{lstlisting}
\end{latin}

پس از بررسی هر موقعییت نتیجه‌ی آن در
\verb;preCalc;
ذخیره می‌شود و در حین بررسی هر موقعییت، اول چک می‌شود که آیا این موقعییت قبلا دیده شده یا نه و اگر دیده شده، آیا عمق آن کافی است.

\begin{latin}
\begin{lstlisting}
	if(preCalc[player].find(curHash) != preCalc[player].end()) {
		if(preCalc[player][curHash].depth >= remDepth)
			return preCalc[player][curHash];
	}


\end{lstlisting}
\end{latin}

\section{\lr{\ttfamily{Alpha-Beta Pruning}}}
برای پیاده سازی
\lr{\ttfamily{Alpha-Beta Pruning}}
پس از بررسی هر بچه در یک موقعییت، مقدار آن بچه با بهترین جواب پدر مقایسه شده.

\begin{latin}
\begin{lstlisting}
		if(curPlayer == player) {
			...
			if(~curBestChoice.bestMove.first && val.value > curBestChoice.value) /* alpha-beta pruning */
				return {val.value, remDepth, mv};
		} else {
			...
			if(~curBestChoice.bestMove.first && val.value < curBestChoice.value) /* alpha-beta pruning */
				return {val.value, remDepth, mv};
		}

\end{lstlisting}
\end{latin}

\section{الگوریتم}
در پیاده سازی الگوریتم، ابتدا با
\verb;Zobrist hashing;
 بررسی می‌کنیم که آیا این  موقعییت قبلا دیده‌شده یا نه.
 اگر
 \verb;remDepth;
 که مقدار عمق باقی‌مانده است، صفر بود، 
 \verb;huristic;
 موقعییت کنونی را بر می‌گردانیم.
 سپس از تابع 
 \verb;getPossibleMoves;
 لیستی از حرکات ممکن می‌گیریم.
 در صورتی که هیچ حرکت نداشتیم، اگر بورد پر بود، به  موقعییت
 \verb;Terminal;
 رسیده‌ایم، در غیر اینصورت موقعییت بورد را در صورتی که نوبت حریفمان‌بود محاسبه می‌کنیم.
 سپس به ازای هر حرکت باقی‌مانده موقعییت بورد در صورت انجام آن حرکت را محاسبه می‌کنیم و بهترین موقعییت را پیدا می‌کنیم.
 
 \begin{latin}
 \begin{lstlisting}
 
 \end{lstlisting}
 \end{latin}
 
 \section{اتصال به \lr{\ttfamily{Python}}}
 
 برای اتصال به 
 \verb;Python;
 کد را به یک
 \lr{\ttfamily{Python extension}}
 تبدیل کردیم.
 اطلاعات بیشتر در پاورقی موجود است.
\LTRfootnote{\url{https://docs.python.org/3/extending/extending.html}}
\LTRfootnote{\url{https://docs.python.org/3/extending/building.html}}
\LTRfootnote{\url{https://docs.scipy.org/doc/numpy-1.15.0/reference/c-api.html}}

به این منظور ابتدا باید با تابع زیر اطلاعات ورودی از 
\verb;Python;
را به فرمت
\lr{\ttfamily{C++}}
تبدیل کنیم.
این تابع پس از اجرای الگوریتم، جدول نهایی را به جدول ورودی مقایسه می‌کند تا از باز کردانی صحیح همه‌ی حرکات اطمینان حاصل کند.

 \begin{latin}
 \begin{lstlisting}
 static PyObject* calculateMiniMax(PyObject* self, PyObject* args) {

	char *argnames[] = {"N", "grid", "player", "max_depth", NULL};

	int N, player, max_depth;
	int *gridVal;
	PyObject *gridArg = NULL;
	PyObject *gridArr = NULL;


	if(!PyArg_ParseTuple(args, "iOii", &N, &gridArg, &player, &max_depth)) {
		PyErr_SetString(PyExc_TypeError,
									"Invalid Parameter");
		return NULL;
	}

	// if(!PyArg_ParseTupleAndKeywords(args, kwds, "", argnames, &N, &grid, &player, &max_depth))
	//   return NULL;
	
	// PyArray_FROM_OTF(gridArg, NPY_INT32, NPY_ARRAY_IN_ARRAY);

	gridVal = (int*) PyArray_DATA((PyArrayObject*)gridArg);

	int **ngrid = new int*[N];
	for(int i = 0; i < N; i++)
		ngrid[i] = new int[N];
	setup(N);
	for(int i = 0; i < STATES; i++)
		preCalc[i] = map<unsigned long long int, PreCalcData>();
	moves = stack<ChangedSegment>();
	for(int i = 0; i < N; i++)
		for(int j = 0; j < N; j++)
			ngrid[i][j] = gridVal[i * N + j];

	// cout << player << " " << max_depth<< endl;
	auto res = MiniMax(N, ngrid, generateHash(N, ngrid), max_depth, player, player);

	for(int i = 0; i < N; i++)
		for(int j = 0; j < N; j++)
			if(ngrid[i][j] != gridVal[i * N + j]) {
				cout << "FATAL ERROR" << endl;
				cout << "MOVE SEQUECE WASN'T SYMMETRICAL" << endl;
				for(int I = 0; I < N; I++, cout << endl)
					for(int J = 0; J < N; J++)
						cout << ngrid[I][J] << " ";
				for(int I = 0; I < N; I++, cout << endl)
					for(int J = 0; J < N; J++)
						cout << gridVal[I * N + J] << " ";
				exit(-1);
			}

	PyObject* ret = PyDict_New();

	PyDict_SetItem(ret, PyUnicode_FromString("value"), PyLong_FromLong(res.value));
	PyDict_SetItem(ret, PyUnicode_FromString("depth"), PyLong_FromLong(res.depth));
	PyDict_SetItem(ret, PyUnicode_FromString("move"), PyTuple_Pack(2, 
				PyLong_FromLong(res.bestMove.first), PyLong_FromLong(res.bestMove.second)));

	return ret;
}

 \end{lstlisting}
 \end{latin}
 
 سپس باید این تابع را در
 \verb;Python;
 رجیستر کنیم.

 \begin{latin}
 \begin{lstlisting}
 
static PyMethodDef MiniMaxMethods[] = {
	{"calculateMiniMax", calculateMiniMax, METH_VARARGS, "Calculate Minimax"},
	{NULL, NULL, 0, NULL}
};

static struct PyModuleDef MiniMaxModule = {
	PyModuleDef_HEAD_INIT,
	"MiniMax",   /* name of module */
	NULL, /* module documentation, may be NULL */
	-1,       /* size of per-interpreter state of the module,
							 or -1 if the module keeps state in global variables. */
	MiniMaxMethods
};

PyMODINIT_FUNC
PyInit_MiniMax(void)
{
	return PyModule_Create(&MiniMaxModule);
}


/* This function is for registering python modules */
int main(int argc, char *argv[]) {
	wchar_t *program = Py_DecodeLocale(argv[0], NULL);
	if (program == NULL) {
			fprintf(stderr, "Fatal error: cannot decode argv[0]\n");
			exit(1);
	}

	/* Add a built-in module, before Py_Initialize */
	if (PyImport_AppendInittab("MiniMax", PyInit_MiniMax) == -1) {
			fprintf(stderr, "Error: could not extend in-built modules table\n");
			exit(1);
	}

	/* Pass argv[0] to the Python interpreter */
	Py_SetProgramName(program);

	/* Initialize the Python interpreter.  Required.
		 If this step fails, it will be a fatal error. */
	Py_Initialize();

	PyMem_RawFree(program);
	return 0;
}

 \end{lstlisting}
 \end{latin}
 
 برای کامپایل و نصب از
 \verb;distutils;
 در
 \verb;Python;
 استفاده شده.


\begin{latin}
\begin{python}
from distutils.core import setup, Extension
import os
import shutil
import numpy

os.environ["CC"] = "g++"
os.environ["CXX"] = "g++"


module1 = Extension('MiniMax',
                    sources=['calc_minimax_eng.cpp'],
                    language="c++17",
                    include_dirs=[numpy.get_include()],
                    extra_compile_args=["-std=c++17", "-Wall", "-Wextra"],
                    )

setup(name='MiniMaxCppExtension',
       version='0.1',
       description='Calculates Minimax for othello',
       ext_modules=[module1])
\end{python}
\end{latin}

